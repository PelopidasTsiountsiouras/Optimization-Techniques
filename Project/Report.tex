\documentclass[a4paper,12pt]{article}

% =========================
% Γλώσσα & Γραμματοσειρές
% =========================
\usepackage{fontspec}
\usepackage{polyglossia}

\setmainlanguage{greek}
\setotherlanguage{english}

\setmainfont{Times New Roman}
\setsansfont{Arial}
\setmonofont{Consolas}

% =========================
% Μαθηματικά & Γραφικά
% =========================
\usepackage{amsmath}
\usepackage{amssymb}
\usepackage{graphicx}
\usepackage{float}
\usepackage{caption}
\usepackage{subcaption}

% =========================
% Κώδικας Matlab
% =========================
\usepackage{listings}
\usepackage{xcolor}

\lstset{
	language=Matlab,
	basicstyle=\ttfamily\small,
	keywordstyle=\color{blue},
	commentstyle=\color{gray},
	stringstyle=\color{purple},
	numbers=left,
	numberstyle=\tiny,
	frame=single,
	breaklines=true,
	tabsize=2
}

% =========================
% Layout & Links
% =========================
\usepackage{geometry}
\geometry{margin=2.5cm}

\usepackage{hyperref}
\hypersetup{
	colorlinks=true,
	linkcolor=black,
	urlcolor=blue
}

\begin{document}   
	
	\begin{titlepage}
		\centering
		\vspace*{1cm}
		{\Large \textbf{Τεχνικές Βελτιστοποίησης}} \\
		\vspace{0.5cm}
		{\large Προσέγγιση Συνάρτησης με Γενετικούς Αλγορίθμους}
		\vspace{1.5cm}
		
		\includegraphics[width=0.35\textwidth]{Logo_AUTH.png}
		
		\vspace{2cm}
		\textbf{Ονοματεπώνυμο} \\
		Πελοπίδας-Νικόλαος Τσιούντσιουρας\\
		\textbf{ΑΕΜ} \\
		 11085
		
		\vspace*{1cm}
		\textbf{Υπεύθυνος Καθηγητής}\\
		Ροβιθάκης Γεώργιος
		
		\vfill
		Τμήμα Ηλεκτρολόγων Μηχανικών και Μηχανικών Υπολογιστών\\
		Αριστοτέλειο Πανεπιστήμιο Θεσσαλονίκης\\
		Ακαδημαϊκό Έτος 2025--2026
	\end{titlepage}
	
	\tableofcontents
	\newpage
	
	\section{Εισαγωγή και Περιγραφή Προβλήματος}
	
	Αντικείμενο της παρούσας εργασίας είναι η μελέτη και η υλοποίηση ενός Γενετικού Αλγορίθμου (Genetic Algorithm - GA) με σκοπό τον προσδιορισμό μιας αναλυτικής έκφρασης που προσεγγίζει τη συμπεριφορά ενός στατικού συστήματος δύο εισόδων. Το σύστημα δέχεται ως εισόδους τις μεταβλητές $u_1$ και $u_2$ και παράγει μια έξοδο $y$, η οποία ορίζεται από μια άγνωστη αλλά συνεχή συνάρτηση $y = f(u_1, u_2)$.
	
	Για την αναλυτική περιγραφή της $f$, χρησιμοποιείται ένας γραμμικός συνδυασμός Γκαουσιανών συναρτήσεων (Gaussian Radial Basis Functions). Συγκεκριμένα, χρησιμοποιούνται έως και 15 Γκαουσιανές συναρτήσεις της μορφής:
	
	\begin{equation}
		G(u_{1},u_{2})=e^{-\left(\frac{(u_{1}-c_{1})^{2}}{2\sigma_{1}^{2}}+\frac{(u_{2}-c_{2})^{2}}{2\sigma_{2}^{2}}\right)}
	\end{equation}
	
	Στόχος του Γενετικού Αλγορίθμου είναι η βελτιστοποίηση των παραμέτρων αυτών των συναρτήσεων (κέντρα $c_1, c_2$, διασπορές $\sigma_1, \sigma_2$ και βάρη του γραμμικού συνδυασμού), ώστε να προκύψει μια μαθηματική έκφραση χαμηλής πολυπλοκότητας που να προσεγγίζει με ακρίβεια τα δεδομένα εισόδου-εξόδου.
	
	Σύμφωνα με τις απαιτήσεις του project, η υλοποίηση του αλγορίθμου πραγματοποιήθηκε εξ ολοκλήρου στο περιβάλλον Matlab χωρίς τη χρήση έτοιμων βιβλιοθηκών βελτιστοποίησης. Για την αξιολόγηση του μοντέλου και την παραγωγή των δεδομένων εκπαίδευσης, χρησιμοποιήθηκε η συνάρτηση αναφοράς:
	
	\begin{equation}
		f(u_{1},u_{2})=\sin(u_{1}+u_{2}) \sin(u_{2}^{2})
	\end{equation}
	
	με περιορισμούς στο πεδίο ορισμού $u_{1} \in [-1, 2]$ και $u_{2} \in [-2, 1]$.
	
	\section{Γενετικοί Αλγόριθμοι (Θεωρητικό Υπόβαθρο)}
	
	Η επίλυση προβλημάτων βελτιστοποίησης χωρίζεται στη μαθηματική μοντελοποίηση και στον σχεδιασμό αλγορίθμων που συγκλίνουν αξιόπιστα σε μια ικανοποιητική λύση. Οι Γενετικοί Αλγόριθμοι (ΓΑ) ανήκουν στην κατηγορία των στοχαστικών τεχνικών ολικής αναζήτησης και προσομοιώνουν τη διαδικασία της φυσικής επιλογής και της εξέλιξης των ειδών.
	
	Σε αντίθεση με τις κλασικές μεθόδους (π.χ. Newton, Συζυγών Κατευθύνσεων), οι ΓΑ δεν απαιτούν πληροφορία παραγώγου της αντικειμενικής συνάρτησης και είναι ιδιαίτερα ανθεκτικοί στην παγίδευση σε τοπικά ελάχιστα, γεγονός που τους καθιστά ιδανικούς για τη βελτιστοποίηση των παραμέτρων των Radial Basis Functions (RBFs) στην παρούσα εργασία.
	
	Η λειτουργία του αλγορίθμου βασίζεται στην επαναληπτική εφαρμογή των εξής σταδίων:
	
	\begin{enumerate}
		\item \textbf{Κωδικοποίηση και Πληθυσμός:} Κάθε υποψήφια λύση κωδικοποιείται σε μια δομή δεδομένων που ονομάζεται \textit{χρωμόσωμα}. Στην περίπτωσή μας χρησιμοποιείται πραγματική κωδικοποίηση (real-valued encoding), όπου κάθε χρωμόσωμα περιλαμβάνει τα βάρη, τα κέντρα και τις διασπορές των 15 Γκαουσιανών.
		
		\item \textbf{Συνάρτηση Καταλληλότητας (Fitness Function):} Αποτελεί το μέτρο αξιολόγησης κάθε ατόμου. Στο συγκεκριμένο πρόβλημα, η καταλληλότητα είναι αντιστρόφως ανάλογη του Μέσου Τετραγωνικού Σφάλματος (MSE) μεταξύ της προσέγγισης και της πραγματικής συνάρτησης $f(u_1, u_2)$.
		
		\item \textbf{Επιλογή (Selection):} Η διαδικασία αυτή εξασφαλίζει ότι τα άτομα με υψηλή καταλληλότητα έχουν μεγαλύτερη πιθανότητα να αναπαραχθούν. Χρησιμοποιείται η μέθοδος του \textit{τουρνουά (tournament selection)}, η οποία προσφέρει ρυθμιζόμενη πίεση επιλογής (selection pressure).
		
		\item \textbf{Διασταύρωση (Crossover):} Είναι ο κύριος γενετικός τελεστής αναζήτησης. Συνδυάζει πληροφορία από δύο γονείς για τη δημιουργία απογόνων, επιτρέποντας την εξερεύνηση (exploration) νέων περιοχών στο χώρο των λύσεων.
		
		\item \textbf{Μετάλλαξη (Mutation):} Εισάγει τυχαίες μεταβολές στα γονίδια με μικρή πιθανότητα. Ο ρόλος της είναι καθοριστικός για τη διατήρηση της γενετικής ποικιλομορφίας του πληθυσμού και την αποφυγή της πρόωρης σύγκλισης (premature convergence).
		
		\item \textbf{Ελιτισμός (Elitism):} Μια στρατηγική όπου το καλύτερο άτομο κάθε γενιάς μεταφέρεται αυτούσιο στην επόμενη, διασφαλίζοντας ότι η ποιότητα της βέλτιστης λύσης που έχει βρεθεί δεν θα μειωθεί κατά τη διαδικασία της εξέλιξης.
	\end{enumerate}
	
	Η επιλογή των υπερπαραμέτρων (μέγεθος πληθυσμού, πιθανότητες μετάλλαξης/διασταύρωσης) αποτελεί κρίσιμο στάδιο, καθώς καθορίζει την ισορροπία μεταξύ της εντατικής εκμετάλλευσης (exploitation) των γνωστών καλών λύσεων και της εξερεύνησης (exploration) του χώρου αναζήτησης.
	
	\section{Υλοποίηση στο Matlab}
	Σε αυτή την ενότητα, θα αναλυθεί η προγραμματιστική υλοποίηση του Γενετικού Αλγορίθμου στο περιβάλλον Matlab, η οποία πραγματοποιήθηκε εξ ολοκλήρου χωρίς τη χρήση έτοιμων συναρτήσεων βιβλιοθήκης. Η υλοποίηση βασίζεται σε μια σπονδυλωτή αρχιτεκτονική (modular architecture), όπου ο κεντρικός αλγόριθμος (\texttt{main.m}) συντονίζει τις εξειδικευμένες συναρτήσεις για τον υπολογισμό, την επιλογή, τη διασταύρωση και τη μετάλλαξη. Ιδιαίτερη έμφαση δόθηκε στη μαθηματική πιστότητα της προσέγγισης μέσω 15 Γκαουσιανών συναρτήσεων και στην ορθή διαχείριση των περιορισμών του προβλήματος, ώστε να επιτευχθεί μια αναλυτική έκφραση χαμηλής πολυπλοκότητας που να ανταποκρίνεται στις απαιτήσεις του συστήματος.
	
	\subsection{Υπολογισμός Εξόδου και Σφάλματος (gaussianCalc.m)}

	Η συνάρτηση \texttt{gaussianCalc.m} αποτελεί το λειτουργικό κέντρο του αλγορίθμου, καθώς υλοποιεί την αναλυτική έκφραση της προσέγγισης για κάθε δεδομένο ζεύγος εισόδων $(u_1, u_2)$.
	
	Το χρωμόσωμα εισέρχεται στη συνάρτηση ως ένα διάνυσμα 75 στοιχείων. Μέσω ενός βρόχου επανάληψης (\texttt{for}), το διάνυσμα "τεμαχίζεται" ανά 5 στοιχεία, ανακτώντας τις παραμέτρους για καθεμία από τις 15 Γκαουσιανές συναρτήσεις:
	\begin{itemize}
		\item \textbf{Βάρος ($w_i$):} Καθορίζει τη συμβολή της $i$-οστής Γκαουσιανής στο τελικό άθροισμα.
		\item \textbf{Κέντρα ($c_{1,i}, c_{2,i}$):} Ορίζουν τη θέση της συνάρτησης στο δισδιάστατο επίπεδο εισόδων.
		\item \textbf{Διασπορές ($\sigma_{1,i}, \sigma_{2,i}$):} Ρυθμίζουν το εύρος επιρροής (width) της συνάρτησης σε κάθε άξονα.
	\end{itemize}
	
	Για κάθε μία από τις $N=15$ συναρτήσεις, υπολογίζεται η τιμή $G_i(u_1, u_2)$ σύμφωνα με τον τύπο:
	\begin{equation}
		G_i(u_1, u_2) = \exp \left( - \left[ \frac{(u_1 - c_{1,i})^2}{2\sigma_{1,i}^2} + \frac{(u_2 - c_{2,i})^2}{2\sigma_{2,i}^2} \right] \right)
	\end{equation}
	
	Η τελική έξοδος του συστήματος προκύπτει από τον γραμμικό συνδυασμό των επιμέρους αποτελεσμάτων:
	\begin{equation}
		\hat{y} = \sum_{i=1}^{15} w_i \cdot G_i(u_1, u_2)
	\end{equation}

	Η υλοποίηση χρησιμοποιεί συσσωρευτή (\texttt{result}) για το άθροισμα, διασφαλίζοντας χαμηλή υπολογιστική πολυπλοκότητα. Η συγκεκριμένη δομή επιτρέπει στον Γενετικό Αλγόριθμο να "μετακινεί" τις Γκαουσιανές στον χώρο και να μεταβάλλει το σχήμα τους, ώστε να ελαχιστοποιηθεί το σφάλμα προσέγγισης σε σχέση με την πραγματική συνάρτηση $f(u_1, u_2) = \sin(u_1+u_2) \sin(u_2^2)$.
	
	\subsection{Διαδικασία Επιλογής (selection.m)}
	
	Η επιλογή των ατόμων που θα στελεχώσουν την επόμενη γενιά (mating pool) αποτελεί έναν από τους κρισιμότερους τελεστές ενός ΓΑ, καθώς καθορίζει την «πίεση επιλογής» (selection pressure). Στην παρούσα υλοποίηση χρησιμοποιήθηκε η μέθοδος της \textbf{επιλογής τουρνουά (Tournament Selection)} με μέγεθος $k=3$.
	
	Για κάθε θέση στον νέο πληθυσμό, η συνάρτηση εκτελεί την εξής διαδικασία:
	\begin{itemize}
		\item Επιλέγονται τυχαία τρία άτομα ($k_1, k_2, k_3$) από τον τρέχοντα πληθυσμό με χρήση της συνάρτησης \texttt{randi(N)}.
		\item Συγκρίνονται οι τιμές σφάλματος (MSE) των τριών αυτών ατόμων, οι οποίες έχουν υπολογιστεί προηγουμένως στη \texttt{main.m}.
		\item Το άτομο με το ελάχιστο σφάλμα (\texttt{min(competitors)}) αναδεικνύεται «νικητής» και αντιγράφεται στον νέο πληθυσμό \texttt{new\_data}.
	\end{itemize}
	
	Όπως αναφέρεται στις τεχνικές βελτιστοποίησης, η επιλογή τουρνουά προτιμάται συχνά έναντι της επιλογής αναλογικής καταλληλότητας (Roulette Wheel Selection) για τους εξής λόγους:
	\begin{enumerate}
		\item \textbf{Αποφυγή Πρόωρης Σύγκλισης:} Δεν βασίζεται στις απόλυτες τιμές του σφάλματος αλλά στη μεταξύ τους κατάταξη, εμποδίζοντας υπερβολικά κυρίαρχα άτομα από το να καταλάβουν αμέσως όλο τον πληθυσμό.
		\item \textbf{Ρυθμισιμότητα:} Το μέγεθος του τουρνουά ($k=3$) επιτρέπει μια μέση πίεση επιλογής, διατηρώντας τη γενετική ποικιλομορφία ενώ ταυτόχρονα οδηγεί τον πληθυσμό προς το ολικό ελάχιστο.
		\item \textbf{Υπολογιστική Αποδοτικότητα:} Η πολυπλοκότητα της διαδικασίας είναι γραμμική $O(N)$, γεγονός που καθιστά τον αλγόριθμο γρήγορο ακόμη και για μεγαλύτερους πληθυσμούς.
	\end{enumerate}
	
	\subsection{Τελεστής Διασταύρωσης (crossover.m)}
	
	Ο τελεστής της διασταύρωσης αποτελεί τον κύριο μηχανισμό αναζήτησης του Γενετικού Αλγορίθμου. Στην παρούσα εργασία υλοποιήθηκε η μέθοδος της \textbf{διασταύρωσης ενός σημείου (single-point crossover)} σε χρωμοσώματα πραγματικών τιμών.
	
	Η συνάρτηση \texttt{crossover.m} δέχεται ως είσοδο το σύνολο των γονέων (mating pool) και την πιθανότητα διασταύρωσης (\texttt{crossover\_probability}), η οποία έχει οριστεί στην \texttt{main.m} ως $p_c = 0.8$. Η διαδικασία ακολουθεί τα εξής βήματα:
	\begin{itemize}
		\item Ο πληθυσμός εξετάζεται ανά ζεύγη γονέων ($i$ και $i+1$).
		\item Για κάθε ζεύγος, παράγεται ένας τυχαίος αριθμός $k \in [0, 1]$.
		\item Εάν $k \leq p_c$, επιλέγεται ένα τυχαίο σημείο τομής $cp$ εντός του εύρους του χρωμοσώματος (από 1 έως 74).
		\item Οι δύο απόγονοι προκύπτουν από την ανταλλαγή των τμημάτων των γονέων μετά το σημείο $cp$:
		\begin{itemize}
			\item \textbf{Απόγονος 1:} Γενετικό υλικό Γονέα 1 έως το $cp$ και Γονέα 2 από το $cp+1$ έως το τέλος.
			\item \textbf{Απόγονος 2:} Γενετικό υλικό Γονέα 2 έως το $cp$ και Γονέα 1 από το $cp+1$ έως το τέλος.
		\end{itemize}
		\item Εάν $k > p_c$, η διασταύρωση δεν πραγματοποιείται και οι γονείς μεταφέρονται αυτούσιοι στην επόμενη γενιά.
	\end{itemize}
	
	Ο τελεστής αυτός επιτρέπει την «εξερεύνηση» (exploration) του χώρου των λύσεων μέσω του επανασυνδυασμού επιτυχημένων «δομικών μονάδων» (schemata). Στη δική μας περίπτωση, εφόσον το χρωμόσωμα αποτελείται από 75 γονίδια (15 Γκαουσιανές $\times$ 5 παράμετροι), η διασταύρωση ενός σημείου επιτρέπει τη διατήρηση της δομής ολόκληρων Γκαουσιανών συναρτήσεων που έχουν ήδη προσαρμοστεί καλά σε κάποια περιοχή της συνάρτησης $f(u_1, u_2)$, ενώ παράλληλα δοκιμάζει νέους συνδυασμούς αυτών. 
	
	Η υψηλή πιθανότητα διασταύρωσης ($80\%$) εξασφαλίζει ότι ο αλγόριθμος θα ανανεώνει συνεχώς το γενετικό υλικό του πληθυσμού, αποφεύγοντας τη στασιμότητα.
	
	\subsection{Τελεστής Μετάλλαξης και Επιβολή Περιορισμών (mutation.m)}
	
	Ο τελεστής της μετάλλαξης είναι απαραίτητος για τη διατήρηση της γενετικής ποικιλομορφίας και την αποφυγή της πρόωρης σύγκλισης σε τοπικά ελάχιστα. Στην παρούσα υλοποίηση, χρησιμοποιήθηκε προσθετική μετάλλαξη με βάση την κανονική κατανομή (Gaussian Mutation), η οποία εφαρμόζεται σε επίπεδο γονιδίου.
	
	Για κάθε άτομο του πληθυσμού και για κάθε γονίδιο ξεχωριστά, εξετάζεται η πιθανότητα μετάλλαξης $p_m = 0.05$. Εάν η συνθήκη ικανοποιείται, η τιμή του γονιδίου μεταβάλλεται σύμφωνα με τη σχέση:
	\begin{equation}
		x_{new} = x_{old} + \text{mutation\_step} \cdot \mathcal{N}(0, 1)
	\end{equation}
	όπου \texttt{mutation\_step} ορίζεται ως $0.1$ και $\mathcal{N}(0, 1)$ είναι μια τυχαία μεταβλητή από την πρότυπη κανονική κατανομή.
	
	Ένα κρίσιμο χαρακτηριστικό της \texttt{mutation.m} είναι η ενσωμάτωση της γνώσης του προβλήματος μέσω της επιβολής περιορισμών. Επειδή το χρωμόσωμα έχει περιοδική δομή ανά 5 γονίδια ($w, c_1, c_2, \sigma_1, \sigma_2$), ο αλγόριθμος αναγνωρίζει τον τύπο της παραμέτρου (\texttt{param\_type}) και επιβάλλει τα αντίστοιχα όρια:
	
	\begin{itemize}
		\item \textbf{Κέντρα $c_1, c_2$:} Διασφαλίζεται ότι οι παράμετροι παραμένουν εντός του πεδίου ορισμού των δεδομένων εκπαίδευσης, δηλαδή $c_1 \in [-1, 2]$ και $c_2 \in [-2, 1]$.
		\item \textbf{Διασπορές $\sigma_1, \sigma_2$:} Επιβάλλεται αυστηρά η συνθήκη $\sigma > 0$. Αυτό είναι απαραίτητο για την αποφυγή μαθηματικών αστοχιών (διαίρεση με το μηδέν) κατά τον υπολογισμό της εκθέτης στην \texttt{gaussianCalc.m}. Σε περίπτωση αρνητικής τιμής, η παράμετρος επαναφέρεται στην τιμή $0.01$.
	\end{itemize}

	Η επιλογή της προσθετικής μετάλλαξης επιτρέπει στον αλγόριθμο να εκτελεί «λεπτομερή ρύθμιση» (fine-tuning) των παραμέτρων. Ενώ η διασταύρωση κάνει μεγάλα άλματα στο χώρο αναζήτησης, η μετάλλαξη επιτρέπει στις Γκαουσιανές συναρτήσεις να προσαρμόζουν το σχήμα και τη θέση τους με ακρίβεια πάνω στις καμπυλότητες της $f(u_1, u_2)$.
	
	\subsection{Κεντρικός Αλγόριθμος και Ροή Εκτέλεσης (main.m)}
	
	Το αρχείο \texttt{main.m} αποτελεί το κύριο πρόγραμμα στο οποίο υλοποιείται η λογική του Γενετικού Αλγορίθμου και η διασύνδεση των επιμέρους συναρτήσεων. Η δομή του ακολουθεί τις βέλτιστες πρακτικές σχεδίασης εξελικτικών αλγορίθμων, διαχωρίζοντας τη διαδικασία σε διακριτές φάσεις.
	
	Αρχικά, ορίζονται οι σταθερές του προβλήματος και οι υπερπαράμετροι του GA:
	\begin{itemize}
		\item \textbf{Πληθυσμός ($N=100$):} Ένας επαρκής αριθμός ατόμων για την κάλυψη του χώρου αναζήτησης 75 διαστάσεων.
		\item \textbf{Διαχωρισμός Δεδομένων:} Παράγονται 500 τυχαία σημεία για την εκπαίδευση (training set) και 500 διαφορετικά σημεία για την αξιολόγηση (validation set), σύμφωνα με την απαίτηση της εκφώνησης για χρήση διαφορετικού συνόλου δεδομένων στη φάση της αξιολόγησης.
		\item \textbf{Αρχικοποίηση:} Ο πληθυσμός αρχικοποιείται τυχαία, με τα κέντρα και τις διασπορές να περιορίζονται εντός των ορίων του πεδίου ορισμού για την επιτάχυνση της σύγκλισης.
	\end{itemize}
	
	Ο αλγόριθμος εκτελείται επαναληπτικά μέχρι να ικανοποιηθεί το κριτήριο τερματισμού (MSE < 0.001) ή να συμπληρωθούν 1000 γενιές. Σε κάθε γενιά εκτελούνται τα εξής:
	\begin{enumerate}
		\item \textbf{Evaluation:} Υπολογίζεται το Μέσο Τετραγωνικό Σφάλμα (MSE) για κάθε χρωμόσωμα χρησιμοποιώντας τη συνάρτηση \texttt{gaussianCalc.m}.
		\item \textbf{Track Best \& Elitism:} Εντοπίζεται το βέλτιστο άτομο της τρέχουσας γενιάς. Εφαρμόζεται στρατηγική \textbf{ελιτισμού}, όπου το καλύτερο χρωμόσωμα που έχει βρεθεί ποτέ (\texttt{best\_chromosome}) τοποθετείται πάντα στην πρώτη θέση του νέου πληθυσμού, διασφαλίζοντας ότι η ποιότητα της λύσης είναι μονοτονικά βελτιούμενη.
		\item \textbf{Evolutionary Steps:} Καλλούνται διαδοχικά οι συναρτήσεις \texttt{selection}, \texttt{crossover} και \texttt{mutation} για τη δημιουργία της επόμενης γενιάς.
	\end{enumerate}

	Μετά τον τερματισμό του βρόχου, το βέλτιστο χρωμόσωμα αξιολογείται στο \textit{unseen data} (validation set) για την επαλήθευση της γενικευτικής ικανότητας του μοντέλου. Τέλος, παράγονται γραφήματα που απεικονίζουν:
	\begin{itemize}
		\item Την καμπύλη μάθησης (Learning Curve), δηλαδή την εξέλιξη του σφάλματος MSE ανά γενιά.
		\item Την τρισδιάστατη επιφάνεια της πραγματικής συνάρτησης έναντι της προσέγγισης που πέτυχε ο GA.
	\end{itemize}
	
	\section{Αποτελέσματα και Αξιολόγηση}
	
	Στην παρούσα ενότητα παρατίθενται τα αποτελέσματα της εξελικτικής διαδικασίας και η αξιολόγηση της τελικής αναλυτικής έκφρασης στο σύνολο δεδομένων ελέγχου (validation set).
	
	Η πορεία της βελτιστοποίησης απεικονίζεται στην Καμπύλη Μάθησης. Παρατηρείται ότι ο αλγόριθμος επιτυγχάνει ταχύτατη μείωση του σφάλματος MSE στις πρώτες 100 γενιές. Στη συνέχεια, ο ρυθμός βελτίωσης μειώνεται, καθώς ο GA εισέρχεται στη φάση της «λεπτομερούς ρύθμισης» (fine-tuning) των παραμέτρων, όπου ο ελιτισμός διασφαλίζει τη διατήρηση των βέλτιστων δομών.
	
	\begin{figure}[H]
		\centering
		\includegraphics[width=0.7\textwidth]{Figure_1.png}
		\caption{Καμπύλη Μάθησης: Εξέλιξη του MSE ανά γενιά.}
	\end{figure}
	
	Στο Σχήμα 2 παρουσιάζεται η σύγκριση μεταξύ της πραγματικής συνάρτησης $f(u_1, u_2)$ και της προσέγγισης που παρήγαγε ο GA. Είναι εμφανές ότι το μοντέλο των 15 Γκαουσιανών κατάφερε να αναπαράγει με υψηλή πιστότητα τη γεωμετρία της επιφάνειας, συλλαμβάνοντας τόσο τα τοπικά μέγιστα όσο και τις περιοχές των ελάχιστων τιμών.
	
	\begin{figure}[H]
		\centering
		\includegraphics[width=0.9\textwidth]{Figure_2.png}
		\caption{Σύγκριση Πραγματικής Συνάρτησης (αριστερά) και Προσέγγισης GA (δεξιά).}
	\end{figure}
	
	Η τελική αναλυτική έκφραση αξιολογήθηκε σε ένα ανεξάρτητο σύνολο 500 σημείων, επιτυγχάνοντας Μέσο Τετραγωνικό Σφάλμα \textbf{MSE = 0.006591}. Η τιμή αυτή υποδηλώνει εξαιρετική γενικευτική ικανότητα, καθώς το σφάλμα παραμένει πολύ χαμηλό σε δεδομένα που ο αλγόριθμος δεν «είδε» κατά την εκπαίδευση.
	
	Στον παρακάτω πίνακα παρουσιάζονται οι βέλτιστες παράμετροι για καθεμία από τις 15 Γκαουσιανές συναρτήσεις:
	
	\begin{table}[H]
		\centering
		\caption{Παράμετροι της Τελικής Αναλυτικής Έκφρασης}
		\small
		\begin{tabular}{|c|c|c|c|c|c|}
			\hline
			\textbf{Gauss} & \textbf{Weight ($w$)} & \textbf{Center 1 ($c_1$)} & \textbf{Center 2 ($c_2$)} & \textbf{Spread 1 ($\sigma_1$)} & \textbf{Spread 2 ($\sigma_2$)} \\ \hline
			1 & 1.1104 & 0.7679 & 1.0000 & 0.7641 & 0.5407 \\ \hline
			2 & -1.7247 & 1.6573 & -0.0474 & 0.0091 & 0.3024 \\ \hline
			3 & 0.4582 & -0.6145 & -1.9449 & 1.0919 & 0.4441 \\ \hline
			4 & 0.3864 & 0.0410 & -0.8876 & 0.3699 & 0.7647 \\ \hline
			5 & -1.2395 & -0.0995 & -1.4131 & 0.8940 & 0.3375 \\ \hline
			6 & -0.4835 & 0.9901 & -0.3771 & 0.4726 & 1.3114 \\ \hline
			7 & 1.0632 & 0.2843 & -1.3275 & 0.8626 & 0.8072 \\ \hline
			8 & 0.3992 & 0.9024 & -1.9773 & 0.3973 & 0.2126 \\ \hline
			9 & -1.3837 & 0.4310 & -1.0383 & 0.7068 & 0.4679 \\ \hline
			10 & 0.3508 & -0.9752 & -2.0000 & 0.5413 & 0.0100 \\ \hline
			11 & 0.7475 & -0.8371 & 0.1474 & 0.4980 & 0.6950 \\ \hline
			12 & -0.7281 & 1.4993 & -1.6919 & 0.8745 & 0.2709 \\ \hline
			13 & 0.9992 & 0.8115 & -0.8518 & 0.4311 & 0.5580 \\ \hline
			14 & -0.9120 & -0.8021 & -0.2565 & 1.0786 & 0.9854 \\ \hline
			15 & 0.8515 & 1.7769 & -1.3096 & 0.6297 & 0.5976 \\ \hline
		\end{tabular}
	\end{table}
	
	\section{Συμπεράσματα και Παρατηρήσεις}
	
	Η ολοκλήρωση του παρόντος project επέτρεψε την εξαγωγή σημαντικών συμπερασμάτων σχετικά με τη χρήση των Γενετικών Αλγορίθμων (ΓΑ) στην προσέγγιση άγνωστων μη γραμμικών συστημάτων.Η υλοποίηση ενός γραμμικού συνδυασμού 15 Γκαουσιανών συναρτήσεων αποδείχθηκε επαρκής για την αναπαράσταση της συνάρτησης στόχου $f(u_1, u_2) = \sin(u_1+u_2) \sin(u_2^2)$.
	
	\begin{itemize}
		\item \textbf{Σύγκλιση και Ακρίβεια:} Ο αλγόριθμος επέδειξε σταθερή σύγκλιση, επιτυγχάνοντας ένα MSE της τάξης του $0.006591$ στο σύνολο δεδομένων αξιολόγησης. Η επιτυχής γενίκευση σε δεδομένα που δεν χρησιμοποιήθηκαν κατά την εκπαίδευση επιβεβαιώνει ότι το μοντέλο δεν υπέστη υπερπροσαρμογή (overfitting).
		\item \textbf{Στρατηγική Ελιτισμού:} Η διατήρηση του βέλτιστου χρωμοσώματος σε κάθε γενιά ήταν καθοριστική. Χωρίς τον ελιτισμό, οι στοχαστικές μεταλλάξεις θα μπορούσαν να οδηγήσουν σε προσωρινή υποβάθμιση της ποιότητας της λύσης, καθυστερώντας τη συνολική σύγκλιση.
		\item \textbf{Διαχείριση Περιορισμών:} Η ενσωμάτωση των περιορισμών απευθείας στον τελεστή της μετάλλαξης διασφάλισε ότι οι παράμετροι των συναρτήσεων βάσης (RBF) παρέμειναν εντός των φυσικών ορίων του προβλήματος, αποφεύγοντας μαθηματικές αστοχίες όπως η διαίρεση με το μηδέν.
	\end{itemize}
	
	Η πολυπλοκότητα της προτεινόμενης αναλυτικής έκφρασης παραμένει χαμηλή, καθώς αποτελείται από 15 όρους, ικανοποιώντας την απαίτηση της εκφώνησης για μια «χαμηλής πολυπλοκότητας» λύση. Η μεθοδολογία αυτή αναδεικνύει τους Γενετικούς Αλγορίθμους ως ένα ισχυρό εργαλείο βελτιστοποίησης σε προβλήματα όπου η αναλυτική μορφή της συνάρτησης κόστους είναι άγνωστη ή ιδιαίτερα σύνθετη.
	
	Συνοψίζοντας, η επιτυχής προσέγγιση της επιφάνειας (Σχήμα 2) και το χαμηλό σφάλμα αξιολόγησης καταδεικνύουν ότι ο σχεδιασμένος Γενετικός Αλγόριθμος είναι ικανός να επιλύει προβλήματα παλινδρόμησης και μοντελοποίησης συστημάτων με υψηλή ακρίβεια και αποδοτικότητα.
	
\end{document}